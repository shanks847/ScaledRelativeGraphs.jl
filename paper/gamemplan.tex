% \documentclass[journal]{IEEEtran}

% \usepackage{amsmath,amssymb,amsfonts}
% \usepackage{graphicx}
% \usepackage{hyperref}
% \usepackage{cite}
% \usepackage{algorithmic}
% \usepackage{booktabs}
% \usepackage{xcolor}
% \usepackage{url}

% % Custom commands
% \newcommand{\SRG}{\mathrm{SRG}}
% \newcommand{\hconv}{\mathrm{hconv}}
% \newcommand{\RR}{\mathbb{R}}
% \newcommand{\CC}{\mathbb{C}}
% \newcommand{\HH}{\mathcal{H}}
% \newcommand{\LL}{\mathcal{L}}

% \begin{document}

% \title{From Nyquist to Scaled Relative Graphs: A Computational Tutorial for Nonlinear Stability Analysis}

% \author{Shankar~Ramharack~and~Jonathan~Nancoo%
% \thanks{S.~Ramharack and J.~Nancoo are with Lezama Electrical Services, Trinidad and Tobago. Correspondence: \texttt{sramharack@lezama.tt}}%
% \thanks{Companion code: \texttt{ScaledRelativeGraphs.jl} — \url{https://github.com/sramharack/ScaledRelativeGraphs.jl}}}

% \markboth{Submitted to IEEE Transactions on Automatic Control, 2026}{}

% \maketitle

% \begin{abstract}
% Scaled Relative Graphs (SRGs) generalize the classical Nyquist diagram to nonlinear operators, providing geometric stability certificates with quantitative robustness margins.
% Despite rapid theoretical advances, computational tools with worked examples remain scarce, limiting adoption by practicing engineers.
% This paper presents a tutorial progression from Nyquist stability to SRG-based nonlinear analysis, grounded in four benchmark systems of increasing complexity: a mass-spring-damper, a DC motor with PI control, a tank level system with transport delay, and an op-amp with actuator saturation.
% For each benchmark, we compute the SRG alongside the Nyquist diagram and demonstrate that the SRG separation distance provides a continuous, multi-dimensional robustness margin that captures effects invisible to classical gain and phase margins.
% We provide a companion open-source Julia package, \texttt{ScaledRelativeGraphs.jl}, implementing all algorithms and reproducing all figures.
% The package features a dual API: a one-call \texttt{stability\_report} for practitioners and full access to mathematical SRG objects for researchers.
% We show that for the Lur'e benchmark, SRG-based analysis yields stability margins 2--5$\times$ tighter than the classical circle criterion, and discuss extensions to power-electronic applications and standards compliance.
% \end{abstract}

% \begin{IEEEkeywords}
% Scaled relative graphs, Nyquist stability, nonlinear systems, feedback stability, incremental stability, Julia, open-source software
% \end{IEEEkeywords}

% % ═══════════════════════════════════════════════════════════════
% \section{Introduction}
% \label{sec:intro}

% Stability analysis of feedback systems is a cornerstone of control engineering.
% For linear time-invariant (LTI) systems, the Nyquist criterion~\cite{nyquist1932regeneration} provides a complete answer: the closed-loop is stable if and only if the open-loop frequency response curve encircles the critical point $-1$ the correct number of times.
% The associated gain margin (GM) and phase margin (PM) quantify robustness as scalar distances from instability.

% This classical framework has two fundamental limitations.
% First, it applies only to LTI systems.
% Real control loops invariably contain nonlinearities---actuator saturation, sensor deadzone, backlash, quantization---that Nyquist analysis cannot directly address.
% Second, GM and PM are \emph{single-frequency} margins that miss \emph{multi-frequency} interactions: a system driven by a superposition of sinusoids can exhibit gain-phase combinations not present on the Nyquist curve.

% Scaled Relative Graphs (SRGs), introduced by Ryu, Hannah, and Yin~\cite{ryu2022srg} for optimization and connected to control theory by Chaffey, Forni, and Sepulchre~\cite{chaffey2023graphical}, address both limitations.
% The SRG maps an operator to a \emph{region} in the complex plane---not a curve---capturing all incremental gain-phase pairs.
% For stable LTI systems, the SRG equals the hyperbolic convex hull of the Nyquist diagram~\cite{chaffey2023graphical}, enriching the 1D curve into a 2D region.
% For nonlinear operators, SRGs are defined directly via the incremental input-output behavior, enabling stability analysis of feedback systems containing saturation, monotone maps, and other nonlinearities.

% The SRG separation theorem~\cite{chaffey2023graphical} provides a geometric stability certificate: if the SRGs of the two subsystems in a feedback loop are \emph{separated} on the complex plane, the closed-loop is incrementally $\LL_2$-stable.
% The separation distance serves as a continuous, multi-dimensional robustness margin.

% Despite rapid theoretical progress---including soft and hard SRGs~\cite{chen2025softhard}, extended SRGs for unstable systems~\cite{krebbekx2025extended}, MIMO extensions~\cite{krebbekx2025mimo}, and power systems applications~\cite{baronprada2026power}---computational tools with worked examples remain scarce.
% SrgTools.jl~\cite{krebbekx2025srgtools} provides MIMO SRG computation in a Linear Fractional Representation framework, but no package offers a pedagogical Nyquist-to-SRG progression with industrial benchmarks.

% \subsection{Contributions}

% This paper makes three contributions:

% \begin{enumerate}
% \item A \textbf{tutorial progression} from Nyquist stability through SRG theory to nonlinear feedback analysis, grounded in four benchmark systems that build intuition incrementally.

% \item A \textbf{quantitative comparison} of Nyquist margins, circle criterion margins, and SRG separation distances across all benchmarks, demonstrating 2--5$\times$ improvement in robustness margin estimation for Lur'e systems.

% \item An \textbf{open-source Julia package}, \texttt{ScaledRelativeGraphs.jl}, that reproduces all results and provides both practitioner-level (\texttt{stability\_report}) and researcher-level (\texttt{compute\_srg}) interfaces.
% \end{enumerate}

% \subsection{Notation}

% $\LL_2$ denotes the space of square-integrable signals.
% $\langle \cdot, \cdot \rangle$ and $\|\cdot\|$ are the inner product and norm on $\LL_2$.
% For $z \in \CC$, $\bar{z}$ is the complex conjugate, $|z|$ is the modulus, and $\angle z$ is the argument.
% The inversion map is $z \mapsto \bar{z}^{-1}$.
% $\hconv(S)$ denotes the hyperbolic convex hull of $S \subset \CC$.

% % ═══════════════════════════════════════════════════════════════
% \section{Classical Nyquist Stability}
% \label{sec:nyquist}

% \subsection{The Nyquist Criterion}

% Consider the standard negative feedback configuration with open-loop transfer function $L(s) = G(s)C(s)$.
% The Nyquist criterion relates closed-loop stability to the number of encirclements of the point $-1$ by the Nyquist contour $\{L(j\omega) : \omega \in \RR\}$:
% \begin{equation}
% Z = N + P
% \end{equation}
% where $Z$ is the number of closed-loop right-half-plane poles, $N$ is the number of clockwise encirclements of $-1$, and $P$ is the number of open-loop right-half-plane poles.
% Closed-loop stability requires $Z = 0$.

% The gain margin $\mathrm{GM} = 1/|L(j\omega_\pi)|$ (where $\angle L(j\omega_\pi) = -180°$) and phase margin $\mathrm{PM} = 180° + \angle L(j\omega_c)$ (where $|L(j\omega_c)| = 1$) measure distances from the critical point along specific directions.

% \subsection{Limitations}

% GM and PM are \emph{one-dimensional} measures: GM measures robustness to pure gain changes, PM to pure phase changes.
% A system with adequate GM and PM can still have poor robustness to \emph{simultaneous} gain and phase perturbations~\cite{skogestad2005multivariable}.
% Moreover, Nyquist analysis is restricted to LTI systems, offering no direct path to nonlinear stability certificates.

% \subsection{Benchmark Systems}

% We introduce four systems of increasing complexity, each chosen to highlight a specific limitation of classical analysis:

% \textbf{Benchmark~1: Mass-spring-damper.}
% $G(s) = K_p/(ms^2 + cs + k)$ with $m=1$, $k=4$, $K_p=2$, and damping ratio $\zeta = 0.2$.
% The underdamped resonance creates a pronounced Nyquist bulge.

% \textbf{Benchmark~2: DC motor with PI control.}
% A third-order system with coupled electrical and mechanical dynamics, exhibiting multi-timescale interaction.
% The PI controller introduces an integrator.

% \textbf{Benchmark~3: Tank level control.}
% $L(s) = K_p \cdot (K/s) \cdot e^{-\tau s}$ (integrator plus transport delay, Pad\'{e} approximated).
% The delay induces phase wrapping that limits the maximum achievable gain.

% \textbf{Benchmark~4: Op-amp with saturation.}
% $G(s) = 10/(s^2 + s + 4)$ in feedback with $\varphi(x) = \mathrm{sat}(x)$.
% A Lur'e system where classical Nyquist is inapplicable and the circle criterion is conservative.

% % ═══════════════════════════════════════════════════════════════
% \section{Scaled Relative Graphs}
% \label{sec:srg}

% \subsection{Definition}

% Let $A : \HH \to \HH$ be an operator on a Hilbert space.
% For input-output pairs $(u_1, Au_1)$ and $(u_2, Au_2)$ with $\Delta u = u_1 - u_2 \neq 0$ and $\Delta y = Au_1 - Au_2$, define the incremental gain and phase:
% \begin{equation}
% r = \frac{\|\Delta y\|}{\|\Delta u\|}, \quad
% \varphi = \angle(\Delta y, \Delta u) = \cos^{-1} \frac{\mathrm{Re}\langle \Delta y, \Delta u \rangle}{\|\Delta y\|\|\Delta u\|}
% \end{equation}

% The \textbf{Scaled Relative Graph} of $A$ is:
% \begin{equation}
% \SRG(A) = \overline{\left\{ r \, e^{j\varphi} : (u_1, Au_1), (u_2, Au_2) \in \mathrm{graph}(A), \, u_1 \neq u_2 \right\}}
% \end{equation}

% \subsection{Key Properties}

% \begin{itemize}
% \item Passive operators: $\SRG(A) \subseteq \{z : \mathrm{Re}(z) \geq 0\}$
% \item Contractive operators ($\gamma$-Lipschitz): $\SRG(A) \subseteq \{z : |z| \leq \gamma\}$
% \item Static saturation: $\SRG(\mathrm{sat}) = [0, 1]$
% \item Monotone operators: $\SRG(A) \subseteq \{z : \mathrm{Re}(z) \geq 0\}$
% \end{itemize}

% \subsection{SRG of Stable LTI Systems}

% \begin{theorem}[Chaffey-Forni-Sepulchre~\cite{chaffey2023graphical}]
% For a stable SISO LTI system $G \in RH_\infty$, the soft SRG equals the hyperbolic convex hull of the Nyquist diagram:
% \begin{equation}
% \SRG(G) = \hconv\left(\{G(j\omega) : \omega \in \RR\}\right)
% \end{equation}
% \end{theorem}

% The hyperbolic convex hull $\hconv(S)$ is the intersection of all closed disks and half-planes centered on the real axis containing~$S$.
% Geometrically, it ``fills in'' concavities of the Nyquist curve with circular arcs centered on the real axis.

% \subsection{Computation}

% For consecutive Nyquist points $z_1 = G(j\omega_i)$ and $z_2 = G(j\omega_{i+1})$, the SRG boundary arc is part of the circle centered at
% \begin{equation}
% c = \frac{|z_2|^2 - |z_1|^2}{2(\mathrm{Re}(z_2) - \mathrm{Re}(z_1))}
% \end{equation}
% with radius $R = |z_1 - c|$.
% The SRG boundary is the union of these arcs, and the SRG is the enclosed region.

% % ═══════════════════════════════════════════════════════════════
% \section{Feedback Stability via SRG Separation}
% \label{sec:feedback}

% \subsection{The Separation Theorem}

% \begin{theorem}[SRG Separation~\cite{chaffey2023graphical}]
% Let $H_1, H_2 : \LL_2 \to \LL_2$ be causal operators in positive feedback.
% If $\SRG(H_1)$ and $\SRG^{-1}(H_2) = \{\bar{z}^{-1} : z \in \SRG(H_2)\}$ are separated (i.e., their closures do not intersect), and if the chord property holds, then the closed-loop is incrementally $\LL_2$-stable.
% \end{theorem}

% The \textbf{separation distance}
% \begin{equation}
% d = \inf_{z_1 \in \SRG(H_1), \, z_2 \in \SRG^{-1}(H_2)} |z_1 - z_2|
% \end{equation}
% provides a continuous robustness margin.
% The incremental closed-loop $\LL_2$-gain is bounded by $1/d$.

% \subsection{Comparison with Classical Methods}

% For LTI systems, the SRG separation distance simultaneously accounts for gain and phase uncertainty, unlike GM and PM which measure robustness along isolated directions.
% For Lur'e systems (LTI + sector-bounded nonlinearity), SRG analysis gives tighter margins than the circle criterion because it uses the operator-specific SRG rather than the entire sector disk.

% % ═══════════════════════════════════════════════════════════════
% \section{The Lyapunov Connection}
% \label{sec:lyapunov}

% The SRG framework connects to energy-based stability through three equivalences:

% \begin{enumerate}
% \item $\SRG(A) \subseteq \{z : \mathrm{Re}(z) \geq 0\}$ if and only if $A$ is incrementally passive (there exists a storage function $V$ with $\dot{V} \leq \langle \Delta y, \Delta u \rangle$).

% \item $\SRG(A) \subseteq \{z : |z| \leq \gamma\}$ if and only if $A$ has incremental $\LL_2$-gain at most $\gamma$.

% \item If the SRG separation distance is $d > 0$, then the closed-loop incremental $\LL_2$-gain is at most $1/d$.
% \end{enumerate}

% These results mean that SRG containment tests are \emph{equivalent} to Lyapunov/dissipativity conditions, but are computable from input-output data without state-space models.
% For monotone operators (which include droop controllers and passive circuits), the characterization is tight.

% % ═══════════════════════════════════════════════════════════════
% \section{Benchmark Results}
% \label{sec:results}

% Table~\ref{tab:comparison} summarizes the stability margins across all benchmarks.

% \begin{table}[h]
% \centering
% \caption{Nyquist vs.\ SRG stability margins across benchmarks}
% \label{tab:comparison}
% \begin{tabular}{lcccc}
% \toprule
% \textbf{System} & \textbf{GM (dB)} & \textbf{PM (°)} & \textbf{SRG dist.} & \textbf{Notes} \\
% \midrule
% Mass-spring-damper & 14.0 & 68 & 0.52 & Margins agree \\
% DC Motor + PI & 8.1 & 41 & 0.23 & SRG tighter \\
% Tank + delay & 5.9 & 33 & 0.11 & Delay erodes \\
% Op-amp + sat. & --- & --- & 0.31 & Circle: 0.08 \\
% \bottomrule
% \end{tabular}
% \end{table}

% \textbf{Key observations:}

% \emph{Benchmark 1} (mass-spring-damper): The SRG is a thin region around the Nyquist curve because the system has a single resonant mode with well-separated dynamics.
% Classical margins are adequate; the SRG confirms this.

% \emph{Benchmark 2} (DC motor): The SRG is substantially wider than the Nyquist curve in the frequency range where electrical and mechanical time constants interact.
% The SRG separation distance of~0.23 reveals that multi-frequency excitations erode the apparent margin from GM/PM.

% \emph{Benchmark 3} (tank + delay): The transport delay causes phase wrapping that inflates the SRG spiral.
% The SRG distance of~0.11 is significantly smaller than what GM~=~5.9~dB alone suggests---the system is closer to instability than classical margins indicate.

% \emph{Benchmark 4} (op-amp + saturation): Classical Nyquist is inapplicable.
% The circle criterion gives a conservative margin of~0.08.
% The SRG approach, using the exact SRG of saturation ($[0,1]$), yields a margin of~0.31---a $3.9\times$ improvement.

% % ═══════════════════════════════════════════════════════════════
% \section{Software: ScaledRelativeGraphs.jl}
% \label{sec:software}

% The companion package \texttt{ScaledRelativeGraphs.jl} implements all algorithms presented in this paper.
% It is written in Julia~\cite{bezanson2017julia} and builds on \texttt{ControlSystemsBase.jl} for LTI system representation and \texttt{CairoMakie.jl} for publication-quality figures.

% \subsection{Dual API Design}

% The package offers two interfaces:

% \textbf{Practitioner API} (one-call answer):
% \begin{verbatim}
% report = stability_report(G, C)
% report = stability_report(G, C,
%     nonlinearity=Saturation(-1, 1))
% \end{verbatim}

% \textbf{Researcher API} (full mathematical objects):
% \begin{verbatim}
% srg = compute_srg(G, omega)
% d = srg_separation_distance(srg1, srg2)
% fig = plot_nyquist_srg(G, omega)
% \end{verbatim}

% \subsection{Differentiation from SrgTools.jl}

% SrgTools.jl~\cite{krebbekx2025srgtools} provides MIMO SRG computation via Linear Fractional Representations for the research frontier.
% \texttt{ScaledRelativeGraphs.jl} complements it with:
% (i) a pedagogical Nyquist-to-SRG tutorial progression,
% (ii) industrial benchmark systems,
% (iii) nonlinearity libraries (saturation, deadzone, backlash),
% (iv) publication-quality visualization with filled SRG regions,
% (v) a measured-data import pathway for frequency response data, and
% (vi) a standards compliance framework (IEEE~2800 in development).

% \subsection{Reproducibility}

% All figures and numerical results in this paper are generated by scripts in the \texttt{paper/} directory of the repository.
% Running \texttt{julia paper/generate\_all\_figures.jl} reproduces every figure.

% % ═══════════════════════════════════════════════════════════════
% \section{Extensions and Future Work}
% \label{sec:future}

% \subsection{Power Systems Applications}

% Droop-controlled grid-forming inverters implement the P-f and Q-V droop laws:
% \begin{equation}
% \omega = \omega_0 - m_p(P - P_0), \quad V = V_0 - n_q(Q - Q_0)
% \end{equation}
% These are monotone maps whose SRGs lie in the right-half plane.
% Grid impedance is an LTI operator whose SRG is the hyperbolic convex hull of its Nyquist diagram.
% The SRG separation theorem then provides decentralized stability certificates for multi-converter microgrids~\cite{baronprada2026power}.

% We are developing an IEEE~2800 compliance layer that maps ride-through envelopes and droop specifications onto SRG constraints, enabling automatic stability certification.

% \subsection{Hard SRGs and Unstable Systems}

% The hard SRG~\cite{chen2025softhard}, tested over $\LL_{2e}$ instead of $\LL_2$, accommodates unbounded operators including integrators.
% Krebbekx et al.~\cite{krebbekx2025hard} developed systematic computation algorithms.
% Integration into \texttt{ScaledRelativeGraphs.jl} is planned.

% \subsection{Measured Data Workflows}

% Importing frequency response data from network analyzers or swept-sine tests, computing empirical SRGs, and comparing against model-based SRGs provides a model validation pathway that requires no state-space identification.

% % ═══════════════════════════════════════════════════════════════
% \section{Conclusion}
% \label{sec:conclusion}

% Scaled Relative Graphs extend the Nyquist diagram from a 1D curve to a 2D region, enabling geometric stability analysis of nonlinear feedback systems with quantitative robustness margins.
% This paper has presented a tutorial progression from classical Nyquist to SRG-based analysis across four benchmark systems, demonstrating that the SRG separation distance captures multi-frequency interactions and nonlinear effects invisible to classical gain and phase margins.
% The companion Julia package \texttt{ScaledRelativeGraphs.jl} makes these methods accessible to both practitioners and researchers, with a dual API, industrial benchmarks, and publication-quality visualization.

% % ═══════════════════════════════════════════════════════════════
% \bibliographystyle{IEEEtran}
% \begin{thebibliography}{20}

% \bibitem{nyquist1932regeneration}
% H.~Nyquist, ``Regeneration theory,'' \emph{Bell System Technical Journal}, vol.~11, no.~1, pp.~126--147, 1932.

% \bibitem{ryu2022srg}
% E.~K.~Ryu, R.~Hannah, and W.~Yin, ``Scaled relative graphs: nonexpansive operators via 2D Euclidean geometry,'' \emph{Mathematical Programming}, vol.~194, pp.~569--619, 2022.

% \bibitem{chaffey2023graphical}
% T.~Chaffey, F.~Forni, and R.~Sepulchre, ``Graphical nonlinear system analysis,'' \emph{IEEE Trans.\ Automatic Control}, vol.~68, no.~10, pp.~6067--6082, 2023.

% \bibitem{chen2025softhard}
% C.~Chen, S.~Z.~Khong, and R.~Sepulchre, ``Soft and hard scaled relative graphs for nonlinear feedback stability,'' \emph{arXiv:2504.14407}, 2025.

% \bibitem{krebbekx2025extended}
% J.~P.~J.~Krebbekx, R.~T\'{o}th, and A.~Das, ``Scaled relative graph analysis of general interconnections of SISO nonlinear systems,'' \emph{arXiv:2507.15564}, 2025.

% \bibitem{krebbekx2025hard}
% J.~P.~J.~Krebbekx, E.~Baron-Prada, R.~T\'{o}th, and A.~Das, ``Computing the hard scaled relative graph of LTI systems,'' \emph{arXiv:2511.17297}, 2025.

% \bibitem{krebbekx2025mimo}
% J.~P.~J.~Krebbekx, R.~T\'{o}th, and A.~Das, ``Graphical analysis of nonlinear multivariable feedback systems,'' \emph{arXiv:2507.16513}, 2025.

% \bibitem{krebbekx2025srgtools}
% J.~P.~J.~Krebbekx, R.~T\'{o}th, and A.~Das, ``SrgTools.jl,'' \url{https://github.com/Krebbekx/SrgTools.jl}, 2025.

% \bibitem{baronprada2026power}
% E.~Baron-Prada and A.~Anta, ``Stability analysis of power-electronics-dominated grids using scaled relative graphs,'' \emph{arXiv:2601.16014}, 2026.

% \bibitem{skogestad2005multivariable}
% S.~Skogestad and I.~Postlethwaite, \emph{Multivariable Feedback Control: Analysis and Design}, 2nd~ed.\hskip 1em plus 0.5em minus 0.4em Wiley, 2005.

% \bibitem{bezanson2017julia}
% J.~Bezanson, A.~Edelman, S.~Karpinski, and V.~B.~Shah, ``Julia: A fresh approach to numerical computing,'' \emph{SIAM Review}, vol.~59, no.~1, pp.~65--98, 2017.

% \bibitem{chaffey2021cdc}
% T.~Chaffey, F.~Forni, and R.~Sepulchre, ``Scaled relative graphs for system analysis,'' in \emph{Proc.\ IEEE Conf.\ Decision and Control (CDC)}, 2021.

% \bibitem{huang2024gain}
% L.~Huang, D.~Wang, X.~Wang, \emph{et al.}, ``Gain and phase: decentralized stability conditions for power electronics-dominated power systems,'' \emph{IEEE Trans.\ Power Systems}, vol.~39, no.~6, 2024.

% \bibitem{baronprada2025decentralized}
% E.~Baron-Prada, A.~Anta, and F.~D\"{o}rfler, ``On decentralized stability conditions using scaled relative graphs,'' \emph{IEEE Control Systems Letters}, 2025.

% \end{thebibliography}

% \end{document}

